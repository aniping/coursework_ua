%%%%%%%%%%%%%%%%%%%%%%%%%%%%%%%%%%%%
% LaTeX template for reading report
% Author: Shuo Yang
%%%%%%%%%%%%%%%%%%%%%%%%%%%%%%%%%%%%

\documentclass[11pt]{article}
\usepackage{amsmath,amssymb,epsfig,graphics,hyperref,amsthm,mathtools,enumitem}
\DeclarePairedDelimiter\ceil{\lceil}{\rceil}
\DeclarePairedDelimiter\floor{\lfloor}{\rfloor}

\hypersetup{colorlinks=true}

\setlength{\textwidth}{7in}
\setlength{\topmargin}{-0.575in}
\setlength{\textheight}{9.25in}
\setlength{\oddsidemargin}{-.25in}
\setlength{\evensidemargin}{-.25in}

\reversemarginpar
\setlength{\marginparsep}{-15mm}

\newcommand{\rmv}[1]{}
\newcommand{\bemph}[1]{{\bfseries\itshape#1}}
\newcommand{\N}{\mathbb{N}}
\newcommand{\Z}{\mathbb{Z}}
\newcommand{\imply}{\to}
\newcommand{\bic}{\leftrightarrow}

% Some user defined strings for the homework assignment
%
\def\CourseCode{CS525}
\def\ReportNo{9}
\def\Category{Reading Report}
\def\PaperTitle{Congestion Avoidance and Control}
\def\Author{Shuo Yang}

\begin{document}

\noindent

\CourseCode \hfill \Category

\begin{center}
Reading Report \#\ReportNo\\
Paper: \PaperTitle\\
Student: \Author\\
\end{center}

% A horizontal split line
\hrule\smallskip
\vspace{1.5em}

The design for congestion avoidance and control proposed by the paper
was definitely a good start point to deal with congested conditions on
the Internet and showed very good result. However, the design is not
perfect, it worked well in 1980s does not mean it is also suitable for
the current Internet, which is much larger and more complex. I will
point out some weak assumptions the paper made and also possible
improvements.

\begin{enumerate}
\item It assumes a binary signal of congestion, that is, either
  congested or not. This model is proved to be simple, yet powerful in
  controlling TCP congestion. But my concern is that would assuming
  such a boolean condition lead to possible oscillation between
  heavy traffic load and light traffic load, thus lead to low
  bandwidth utilization and make the Internet unstable? For example,
  when congestion occurs, many end hosts will lose packets and then
  dramatically decrease the TCP windows size, this drops the network
  bandwidth significantly. But then these hosts realize the network is
  under utilization, thus they all increase the window size and make
  the network under full load very soon. This boolean
  condition might model the 1980s' Internet well, but how about the
  current Internet, where the situation is quite complex? It seems to
  me that a fine-grained definition of different levels of TCP
  congestion might be better for today's Internet. Because it can
  adapt to the different congestion levels and achieve better
  bandwidth utilization. 
\item The paper uses packet loss as the indicator of congestion, this
  assumption is reasonable but not always true. For some packets, the
  delay of transmission might be just very long. It is also possible
  that it is hardware failure that lead to packet loss. Thus using
  packet loss to predict the network congestion is somewhat
  biased. But solely relying on the end hosts to avoid and control the
  congestion is indeed very hard. We are again back to the end-to-end
  argument! Should we push congestion control completely to the end
  hosts or also let the network play a role in congestion control?
  End-to-end argument is not a golden principle, it might be better if
  we could strike a balance between the responsibilities of congestion
  control of end hosts and network. Doing this not only helps us
  control the congestion more accurately, but eliminates the fairness
  problem if we only rely on end hosts to control the congestion. For
  example, some malicious end hosts do not play by the rule and try to
  take all the bandwidth and leave the congestion to other users that
  play by the rule. This make the network vulnerable. On the other
  hand, pushing some congestion control functionality into network
  might lead to better QoS since the network can allocate bandwidth 
  more intelligently according to the congestion level it observed. 
\end{enumerate}

\end{document}
