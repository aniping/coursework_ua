%%%%%%%%%%%%%%%%%%%%%%%%%%%%%%%%%%%%
% LaTeX template for reading report
% Author: Shuo Yang
%%%%%%%%%%%%%%%%%%%%%%%%%%%%%%%%%%%%

\documentclass[11pt]{article}
\usepackage{amsmath,amssymb,epsfig,graphics,hyperref,amsthm,mathtools,enumitem}
\DeclarePairedDelimiter\ceil{\lceil}{\rceil}
\DeclarePairedDelimiter\floor{\lfloor}{\rfloor}

\hypersetup{colorlinks=true}

\setlength{\textwidth}{7in}
\setlength{\topmargin}{-0.575in}
\setlength{\textheight}{9.25in}
\setlength{\oddsidemargin}{-.25in}
\setlength{\evensidemargin}{-.25in}

\reversemarginpar
\setlength{\marginparsep}{-15mm}

\newcommand{\rmv}[1]{}
\newcommand{\bemph}[1]{{\bfseries\itshape#1}}
\newcommand{\N}{\mathbb{N}}
\newcommand{\Z}{\mathbb{Z}}
\newcommand{\imply}{\to}
\newcommand{\bic}{\leftrightarrow}

% Some user defined strings for the homework assignment
%
\def\CourseCode{CS525}
\def\ReportNo{11}
\def\Category{Reading Report}
\def\PaperTitle{Development of the Domain Name System}
\def\Author{Shuo Yang}

\begin{document}

\noindent

\CourseCode \hfill \Category

\begin{center}
Reading Report \#\ReportNo\\
Paper: \PaperTitle\\
Student: \Author\\
\end{center}

% A horizontal split line
\hrule\smallskip
\vspace{1.5em}

DNS was designed to address the limitation of using HOSTS.TXT for
name resolution. It is quite clear that DNS
is a better choice than HOSTS.TXT for fast-scaling Internet. However,
it is not clear, according to the paper, why DNS stands out as a better
design compared to other alternatives? The strengths of DNS lie in its
use of distributed database,
hierarchical name space and caching. But there seems to be also some
drawbacks to the design decisions made for DNS.

\vspace{1em}
The Internet was designed to be resilient under failures, but the
design of DNS seems to put single points of failure into the Internet,
that is, the root servers. Though root servers can be replicated, but
what if they all go down? Then we either rely solely on cache or we
get stuck. What if they all contain configuration errors such that we
cannot get a correct IP address for a specific host? Then even the
physical connections are there, we still get stuck. The design of DNS
strikes a balance between centralized and distributed management, a 
more distributed way of managing this may be needed for the Internet
to survive under hard failures (e.g, server goes down) or soft failures
(e.g, configuration errors).

\vspace{1em}
The original DNS design puts a heavily focus on usability, without
mentioning even a little bit about security, which is now a huge
problem for DNS. The well known DNS attack targets exactly those DNS
vulnerabilities. These attacks take advantage of the communication
back and forth between DNS clients and DNS servers. Now we have to put
additional efforts in making DNS lookups secure. But if we could put
security as a primary design principle, we could end up getting less
trouble. The design of DNS assumes mutual trust between clients and
servers. This assumption, though holds for old days, now opens the
door for various of attacks. We need secure protocols for
clients and servers to verify identify and exchange information.

\vspace{1em}
The design of DNS adds a level of indirection, that is, just for name
resolution. This brings concerns of performance. Indeed, nowadays,
often time it is slow DNS lookup that causes the slow service response
rather than other factors. Your requests may go back and forth between many
DNS servers, from root server to local servers at different levels, to
eventually get the results back. The question is: is this extra level
of indirection really needed? Can we perform name resolution at a
higher network layer with better security, protocol and performance
for similar service? I believe answering this question can lead to
better design for Internet name resolution.

\end{document}
