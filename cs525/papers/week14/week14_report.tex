%%%%%%%%%%%%%%%%%%%%%%%%%%%%%%%%%%%%
% LaTeX template for reading report
% Author: Shuo Yang
%%%%%%%%%%%%%%%%%%%%%%%%%%%%%%%%%%%%

\documentclass[11pt]{article}
\usepackage{amsmath,amssymb,epsfig,graphics,hyperref,amsthm,mathtools,enumitem}
\DeclarePairedDelimiter\ceil{\lceil}{\rceil}
\DeclarePairedDelimiter\floor{\lfloor}{\rfloor}

\hypersetup{colorlinks=true}

\setlength{\textwidth}{7in}
\setlength{\topmargin}{-0.575in}
\setlength{\textheight}{9.25in}
\setlength{\oddsidemargin}{-.25in}
\setlength{\evensidemargin}{-.25in}

\reversemarginpar
\setlength{\marginparsep}{-15mm}

\newcommand{\rmv}[1]{}
\newcommand{\bemph}[1]{{\bfseries\itshape#1}}
\newcommand{\N}{\mathbb{N}}
\newcommand{\Z}{\mathbb{Z}}
\newcommand{\imply}{\to}
\newcommand{\bic}{\leftrightarrow}

% Some user defined strings for the homework assignment
%
\def\CourseCode{CS525}
\def\ReportNo{14}
\def\Category{Reading Report}
\def\PaperTitle{Clustering and Sharing Incentives in BitTorrent
  Systems}
\def\Author{Shuo Yang}

\begin{document}

\noindent

\CourseCode \hfill \Category

\begin{center}
Reading Report \#\ReportNo\\
Paper: \PaperTitle\\
Student: \Author\\
\end{center}

% A horizontal split line
\hrule\smallskip
\vspace{1.5em}
This work is an experimental investigation on BitTorrent protocol
properties: the clustering of similar-bandwidth peers, the
effectiveness of BitTorrent's sharing incentives, and the peers' high
uplink utilization. It helps us to better understand BitTorrent
protocol. Based on this work, we can do some followup work on
BitTorrent protocol. We will look at three aspects: performance,
fair sharing and BitTorrent for streaming.

\begin{enumerate}
\item \emph{performance} \\
  One way to improve performance is to explore BitTorrent
  locality. BitTorrent ignores traffic costs at ISPs and generates a
  lot of cross-ISP traffic. This is not desirable. We need to improve
  BitTorrent to enhance its traffic locality,
  trying to push traffic within an ISP and to minimize traffic across
  ISPs. Peers in BitTorrent select their neighbors complete
  randomly. This is the root cause of cross-ISP traffic. If we can
  make a peer select most of its neighbors within the same ISP, then
  we can reduce cross-ISP traffic and lead to better performance. To
  achieve this, BitTorrent needs to be aware of the underlying
  topology at the ISP level, this definitely complicates the
  protocol. Another way to improve performance is to encourage nodes
  to stay as seeds for a longer time. Current BitTorrent lacks incentives for nodes to
  stay as seeds. If nodes can stay around as seeds after finish
  downloading, it can help newly joined nodes catch up faster. Due to
  lack of data pieces they can offer to other nodes, newly joined nodes
  usually need longer time to reach to a fast and stable download
  rate. They usually receives data pieces from optimistic unchoking of other
  leechers as well as from seeds. Therefore having a large number of
  seeds available can allow them ramp up faster. The question of how to
  encourage nodes to stay as seeds remains open-ended.

\item \emph{fair sharing} \\
  BitTorrent has the problem of potentially
  having peers who consume more than their fair share of the network
  resources, the so-called free-riders. We need a way to correctly
  identify free-riders and how to adjust the policy to penalize
  free-riders and reward non-free-riders. We can also do some work to
  investigate how these free-riders impact the system performance.

\item \emph{BitTorrent for streaming} \\
  Though not originally designed for streaming, BitTorrent can be
  potentially used for streaming. Changes are needed for BitTorrent to
  be suitable for streaming. The challenge is to meet the sequential
  playback demands of media streaming applications. 
  One major thing is piece selection: we
  need better selection policy to satisfy streaming.
\end{enumerate}

\vspace{1em}

\end{document}
