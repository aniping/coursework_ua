%%%%%%%%%%%%%%%%%%%%%%%%%%%%%%%%%%%%
% LaTeX template for reading report
% Author: Shuo Yang
%%%%%%%%%%%%%%%%%%%%%%%%%%%%%%%%%%%%

\documentclass[11pt]{article}
\usepackage{amsmath,amssymb,epsfig,graphics,hyperref,amsthm,mathtools,enumitem}
\DeclarePairedDelimiter\ceil{\lceil}{\rceil}
\DeclarePairedDelimiter\floor{\lfloor}{\rfloor}

\hypersetup{colorlinks=true}

\setlength{\textwidth}{7in}
\setlength{\topmargin}{-0.575in}
\setlength{\textheight}{9.25in}
\setlength{\oddsidemargin}{-.25in}
\setlength{\evensidemargin}{-.25in}

\reversemarginpar
\setlength{\marginparsep}{-15mm}

\newcommand{\rmv}[1]{}
\newcommand{\bemph}[1]{{\bfseries\itshape#1}}
\newcommand{\N}{\mathbb{N}}
\newcommand{\Z}{\mathbb{Z}}
\newcommand{\imply}{\to}
\newcommand{\bic}{\leftrightarrow}

% Some user defined strings for the homework assignment
%
\def\CourseCode{CS525}
\def\ReportNo{8}
\def\Category{Reading Report}
\def\PaperTitle{Multicast Routing in Datagram Internetworks and
  Extended LANS}
\def\Author{Shuo Yang}

\begin{document}

\noindent

\CourseCode \hfill \Category

\begin{center}
Reading Report \#\ReportNo\\
Paper: \PaperTitle\\
Student: \Author\\
\end{center}

% A horizontal split line
\hrule\smallskip
\vspace{1.5em}

The paper mainly presents three efficient routing algorithms for
multicasting in the Internet and extended LANs by modifying the existent
unicast protocols such as distance vector and link-state protocols. 

\vspace{1em}
I would like to argue that the design of such multicasting algorithms
kind of violates the end-to-end principle raised in the ``End-to-end
arguments in system design'' paper. The end-to-end principle
states that in a general-purpose network, application-specific
functions ought to reside in the end hosts of a network rather than in
intermediary nodes, given that these functions can be implemented
completely and correctly in the end hosts. 

\vspace{1em}
First, the functionality of one-to-many or many-to-many
delivery over the network can be implemented completely on the end
hosts without resorting to the network itself by using repetitive
unicasting. Doing this will put too much additional traffic to the
network, but it simplifies the network itself and puts the
multicast functionality to the end users, they may implement it in
anyway they want based on the unicast, thus stick to the principle of
end-to-end argument. 

\vspace{1em}
On the other hand, for the algorithms proposed by the paper, the
network itself has to keep some states in order to achieve efficient
multicasting, thus the modified distance vector or link-state
protocols are not stateless protocol anymore. The main trade-off made
by the paper is low delay of joining in or deliverying to a multicast
group vs. increased cost in router state maintenance. For example,

\begin{enumerate}
\item For the basic algorithm used with LAN bridges, a router involved
  in multicasting needs to maintain a table of multicast records and
  periodically increment ages of multicast table records, and expire
  them. This lowers the delay of joining and leaving a multicast
  group, but at the cost of increased bandwidth used for periodic
  membership reporting, which is used to construct the tables, and
  of course, the additional state information the router maintained
  for multicast table records.
\item For the distance vector multicasting, a router needs to deal
  with membership reports and keep track of a single closest
  ``parent'' router for each link relative to a multicasting source
  S. It needs to transit, store and process non-membership reports
  also.
\item For link-state multicasting, a router needs to consider
  multicast membership as part of the link ``state'' as well.
\end{enumerate}

Thus, this paper seems to serve as a counter-argument to the end-to-end
argument. Apparently, multicasting is important and indeed we need
efficient and scalable multicasting protocols. However, completely
replying on end hosts to implement multicasting does not sound as a
good idea. We need strong and robust network design to achieve these
goals.

\end{document}
