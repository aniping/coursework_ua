%%%%%%%%%%%%%%%%%%%%%%%%%%%%%%%%%%%%
% LaTeX template for reading report
% Author: Shuo Yang
%%%%%%%%%%%%%%%%%%%%%%%%%%%%%%%%%%%%

\documentclass[11pt]{article}
\usepackage{amsmath,amssymb,epsfig,graphics,hyperref,amsthm,mathtools,enumitem}
\DeclarePairedDelimiter\ceil{\lceil}{\rceil}
\DeclarePairedDelimiter\floor{\lfloor}{\rfloor}

\hypersetup{colorlinks=true}

\setlength{\textwidth}{7in}
\setlength{\topmargin}{-0.575in}
\setlength{\textheight}{9.25in}
\setlength{\oddsidemargin}{-.25in}
\setlength{\evensidemargin}{-.25in}

\reversemarginpar
\setlength{\marginparsep}{-15mm}

\newcommand{\rmv}[1]{}
\newcommand{\bemph}[1]{{\bfseries\itshape#1}}
\newcommand{\N}{\mathbb{N}}
\newcommand{\Z}{\mathbb{Z}}
\newcommand{\imply}{\to}
\newcommand{\bic}{\leftrightarrow}

% Some user defined strings for the homework assignment
%
\def\CourseCode{CS525}
\def\ReportNo{10}
\def\Category{Reading Report}
\def\PaperTitle{Congestion Control for High Bandwidth-Delay Product
  Networks}
\def\Author{Shuo Yang}

\begin{document}

\noindent

\CourseCode \hfill \Category

\begin{center}
Reading Report \#\ReportNo\\
Paper: \PaperTitle\\
Student: \Author\\
\end{center}

% A horizontal split line
\hrule\smallskip
\vspace{1.5em}

XCP is designed to improve traditionally TCP congestion control
methods, the goal is still the same: to maximum network bandwidth
utilization while ensuring the fairness among flows. The design is
still focused on network itself. If we take a look at the metrics it
used for measuring the effectiveness of XCP: \emph{average link
  utilization}, \emph{average queue size} and \emph{number of packet
  drops} for efficiency and \emph{average throughput per flow} for
fairness, we will notice that these metrics are all
network-centric. But how about the end-user experience? Do they really
feel like they get good services when XCP is in use?

\vspace{1em}
All the benefits XCP brought in also came with a prices. Because of
the ignore of the end-user experience, XCP does not perform well in
any cases. For example, for long-lived flows, XCP performs very well,
being able to achieve high bandwidth utilization, small queue size,
low packet drops and fair bandwidth share among flows. However, the
Internet is a dynamic environment, therefore we always see flows with
different length. Especially, as the long-tail distribution suggested:
there are more short-lived flows and fewer long-lived flows. Under
such condition, will all the performance benefits envisioned by XCP
still hold? 

\vspace{1em}
With a mix of flows of different length, XCP does not perform
well for short flows. In XCP, when there are available bandwidths,
its fairness controller will divide it evenly among flows. This seems
not fair to those short-lived flows because it is possible that they
could have finished within one or few RTTs, but because of the way
XCP's fairness controller works, they still get very little bandwidth
increase for each round trip. This could stretch their duration to
many RTTs, which is definitely not desirable. From end-user point of
view, they might experience relative long service delay than they
should. Things get worse when several long-lived flows keep the
link occupied, when new short flows come in, they will get very little
share of bandwidth, though they will gain bandwidth gradually, since
they are short flows, they may have finished before they enter
fair-share state. Thus XCP leads to long flow duration for those short
flows, this will negatively impact end-user experience of
services through the Internet.

\vspace{1em}
The way XCP controls fairness may be a bit over conservative. The use
of AIMD leads to slow converge to fair-share state of
flows. Ignorance of different flow sizes leads to long duration for
short flows. For new arriving flows, XCP will gradually decrease the
window sizes of existing flows to allocate bandwidth for new flows,
thus it takes time for new flows to get their fair-share of
bandwidth. All these contribute to the unfairness of XCP.
A better congestion control mechanism should be able adjust to short
flows to let them finish early as they should, and be able to allocate
bandwidth to new flows more aggressively then XCP. 

\end{document}
