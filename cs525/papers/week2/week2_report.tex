%%%%%%%%%%%%%%%%%%%%%%%%%%%%%%%%%%%%
% LaTeX template for reading report
% Author: Shuo Yang
%%%%%%%%%%%%%%%%%%%%%%%%%%%%%%%%%%%%

\documentclass[11pt]{article}
\usepackage{amsmath,amssymb,epsfig,graphics,hyperref,amsthm,mathtools,enumitem}
\DeclarePairedDelimiter\ceil{\lceil}{\rceil}
\DeclarePairedDelimiter\floor{\lfloor}{\rfloor}

\hypersetup{colorlinks=true}

\setlength{\textwidth}{7in}
\setlength{\topmargin}{-0.575in}
\setlength{\textheight}{9.25in}
\setlength{\oddsidemargin}{-.25in}
\setlength{\evensidemargin}{-.25in}

\reversemarginpar
\setlength{\marginparsep}{-15mm}

\newcommand{\rmv}[1]{}
\newcommand{\bemph}[1]{{\bfseries\itshape#1}}
\newcommand{\N}{\mathbb{N}}
\newcommand{\Z}{\mathbb{Z}}
\newcommand{\imply}{\to}
\newcommand{\bic}{\leftrightarrow}

% Some user defined strings for the homework assignment
%
\def\CourseCode{CS525}
\def\ReportNo{2}
\def\Category{Reading Report}
\def\PaperTitle{End-to-End Argument in System Design}
\def\Author{Shuo Yang}

\begin{document}

\noindent

\CourseCode \hfill \Category

\begin{center}
Reading Report \#\ReportNo\\
Paper: \PaperTitle\\
Student: \Author\\
\end{center}

% A horizontal split line
\hrule\smallskip
\vspace{1.5em}

The entire paper is based on this line of reasoning: \emph{``The function in
question can completely and correctly be implemented only with the
knowledge and help of the application standing at the end points of
the communication system. Therefore, providing that questioned
function as a feature of the communication system itself is not
possible.''}.

I would argue that this line of reasoning has some weak points which
need to be addressed in order to apply it in reality, especially under
the context of the current Internet.

\begin{enumerate}
\item The paper doesn't say exactly what \emph{complete and
  correct implementation of a function} means. Does it mean the
  fidelity of data being transferred, or integrity, or something else?
  I think it varies by applications. For example, for file transfer
  application, we want to achieve integrity of data, but for video
  conference, we want to ensure that video and audio data be
  streamed in real time and in sync, some loss of quality is
  tolerable. 
\item Further, the paper draws an absolute line for a function in
  question. That is, it can be either implemented ``completely and
  correctly'' or not. This is not true. Actually, for file transfer
  application, reliable file transmission 
  cannot be 100\% guaranteed. For example, even with ``end
  to end checksum of the file'' mentioned in the paper, there is a
  still slight chance that data could be altered in the path from
  source to destination while the destination sees the checksum as
  expected. This is due to the probabilistic nature of the checksum
  algorithm. 
\item The argument is justified mostly by the ``careful file
  transfer'' example used in the paper. But in reality, network
  applications are rich in variety. While \emph{end-to-end argument}
  is effective for bulk data transfer applications like FTP and
  BitTorrent, it may need careful evaluation for applications like
  video streaming or instant messaging. This is especially true when
  considering the current Internet filled with complex and interactive
  multimedia applications with different cost and needs for quality of
  services.
\item Under the context of the current Internet, we need to rethink
  about the idea of ``dumb network'' and ``pushing the logic to the end
  points'' which are advocated by \emph{end-to-end argument}. This
  idea has made the regular home network very complex 
  already. Considering the emerging mobile ad hoc network and Internet
  of Things, 
  they all feature thin client, which means we have to push some functions
  into network itself to make them work properly. These emerging
  networks demand new requirements to Internet, and make the
  \emph{end-to-end argument} not hold under specific context.
\item The paper dose not go deep to give the detailed guidelines of
  how to apply 
  the \emph{end-to-end argument} in practice. Just letting application
  programmers implement complex logic seems a horrible idea. Should
  there be libraries to do these? Should we do these at the kernel space
  or user space? These questions remain relevant in network system
  design.
\end{enumerate}

\emph{Conclusion}: with the changing requirements of the Internet, we
need to 
carefully evaluate and apply the \emph{end-to-end argument} in a
context-sensitive manner, rather than adopting it as a general design
principle/rule for network system.

\end{document}
