%%%%%%%%%%%%%%%%%%%%%%%%%%%%%%%%%%%%
% LaTeX template for reading report
% Author: Shuo Yang
%%%%%%%%%%%%%%%%%%%%%%%%%%%%%%%%%%%%

\documentclass[11pt]{article}
\usepackage{amsmath,amssymb,epsfig,graphics,hyperref,amsthm,mathtools,enumitem}
\DeclarePairedDelimiter\ceil{\lceil}{\rceil}
\DeclarePairedDelimiter\floor{\lfloor}{\rfloor}

\hypersetup{colorlinks=true}

\setlength{\textwidth}{7in}
\setlength{\topmargin}{-0.575in}
\setlength{\textheight}{9.25in}
\setlength{\oddsidemargin}{-.25in}
\setlength{\evensidemargin}{-.25in}

\reversemarginpar
\setlength{\marginparsep}{-15mm}

\newcommand{\rmv}[1]{}
\newcommand{\bemph}[1]{{\bfseries\itshape#1}}
\newcommand{\N}{\mathbb{N}}
\newcommand{\Z}{\mathbb{Z}}
\newcommand{\imply}{\to}
\newcommand{\bic}{\leftrightarrow}

% Some user defined strings for the homework assignment
%
\def\CourseCode{CS525}
\def\ReportNo{12}
\def\Category{Reading Report}
\def\PaperTitle{Chord: A Scalable Peer-to-peer Lookup Protocol for
  Internet Applications}
\def\Author{Shuo Yang}

\begin{document}

\noindent

\CourseCode \hfill \Category

\begin{center}
Reading Report \#\ReportNo\\
Paper: \PaperTitle\\
Student: \Author\\
\end{center}

% A horizontal split line
\hrule\smallskip
\vspace{1.5em}

The strengths of Chord include \emph{highly efficient search},
\emph{good load balancing} and \emph{completely decentralized design}.
I would also like to point out some disadvantages of the design of Chord.

\vspace{1em}
First, Chord does not consider physical location of nodes in their
search ring design. Each node directs its lookup request based on
its routing table (or finger table), without searching for the desired
data in the nodes that are in its geographical range first. This may
increase lookup time and waste network bandwidth significantly, and
make the performance of Chord much less predictable. A node in
Boston may have its successor node in Seattle, where it could have
chosen a node in New York as its successor if Chord took geographic
location into design consideration. To fix this drawback, we could
redesign Chord to considers physical location of nodes present in the
network while still maintaining other Chord's properties in order to
minimize the real distance messages travel. 

\vspace{1em}
Another disadvantage of Chord is its asymmetric lookup, that is,
lookup from a node to another node could take a significant different number of
hops than vice versa. Considering a very large Chord ring where node
$q$ precedes node $p$. If node $p$ wants to look up node $q$, it has
to travel over the complete ring to reach to $p$, while it only takes
one hop for $p$ to reach $q$ if $p$ is able to lookup backwards to its
preceding nodes. To address this inefficiency, we could have two finger
tables, one for looking up succeeding nodes, another for looking up
preceding nodes. In doing this node in the Chord ring will have the
capability of searching both clockwise and counterclockwise. Upon
receiving a lookup request, a node should first determine which
direction produces a shorter path to the target node before sending
out lookup requests.

\vspace{1em}
Because Chord is basically a distributed hash table, it can only
support exact match queries. But in reality, keyword-based queries is
more popular, Chord has no way to support this. Chord also does not
take advantage of the popularity of files being searched. One
possible solution is to rank the popularity of each file and let the
nodes in the ring cache those nodes with highly ranked files to
speed up search. The final disadvantage for Chord is that all nodes
must have IDs, thus Chord does not allow anonymous nodes to join the
ring. This brings concern about security. What if some node does not
want others to know its IP? Chord has no way to hide the identify of
the nodes.

\end{document}
