%%%%%%%%%%%%%%%%%%%%%%%%%%%%%%%%%%%%
% LaTeX template for reading report
% Author: Shuo Yang
%%%%%%%%%%%%%%%%%%%%%%%%%%%%%%%%%%%%

\documentclass[11pt]{article}
\usepackage{amsmath,amssymb,epsfig,graphics,hyperref,amsthm,mathtools,enumitem}
\DeclarePairedDelimiter\ceil{\lceil}{\rceil}
\DeclarePairedDelimiter\floor{\lfloor}{\rfloor}

\hypersetup{colorlinks=true}

\setlength{\textwidth}{7in}
\setlength{\topmargin}{-0.575in}
\setlength{\textheight}{9.25in}
\setlength{\oddsidemargin}{-.25in}
\setlength{\evensidemargin}{-.25in}

\reversemarginpar
\setlength{\marginparsep}{-15mm}

\newcommand{\rmv}[1]{}
\newcommand{\bemph}[1]{{\bfseries\itshape#1}}
\newcommand{\N}{\mathbb{N}}
\newcommand{\Z}{\mathbb{Z}}
\newcommand{\imply}{\to}
\newcommand{\bic}{\leftrightarrow}

% Some user defined strings for the homework assignment
%
\def\CourseCode{CS525}
\def\ReportNo{5}
\def\Category{Reading Report}
\def\PaperTitle{On inferring autonomous system relationships in the Internet}
\def\Author{Shuo Yang}

\begin{document}

\noindent

\CourseCode \hfill \Category

\begin{center}
Reading Report \#\ReportNo\\
Paper: \PaperTitle\\
Student: \Author\\
\end{center}

% A horizontal split line
\hrule\smallskip
\vspace{1.5em}

In the paper, the author proved that the AS path in any BGP routing
table entry is valley-free based on the assumption that all ASs set
their export policies according to the selective export rule. But in
reality, not all ASs stick to the this export policy. Thus, not all AS
paths are valley free. Also, as author mentioned in the paper,
sometimes the top provider does not have the highest degree. Because
of these inaccuracies and also because of BGP data collected by
route-view project is only a partial view of the Internet, it is very
hard to accurately infer AS relationships.

\vspace{0.5em}
However, if we took a different view towards the given BGP data, we
might get some interesting results back. First, we have the data that
reflects the BGP routing of the Internet. Second, we know that
in the ideal situation, all inter-domain routing should be valley
free. Thus, we can explore the BGP data to identify those
non-valley-free BGP paths that violate the agreed upon export
policy. In detail, we would like to know to what extent
non-valley-free BGP paths exist in current Internet. For example, how
many non-valley-free BGP paths were advertised during a fixed time
period? What percentage do they take from the total BGP paths? We also
want to find out what
are the reasons that cause these violations (deliberately or due to
misconfigurations)? Is there any common pattern of violations hidden
behind the scenes? What are the source ASs of these violations?
And further we can evaluate how do these violations impact the Internet
routing? 

\vspace{0.5em}
Answering these questions can benefit many parties with different
interests.
\vspace{0.5em}
\begin{itemize}[nolistsep]
\item Routing policy maker can enforce additional rules to minimize
  such violations.
\item Researchers and organizations can develop better tools for
  detecting and preventing such violations.
\item Researchers and organizations can develop or improve
  inter-domain routing protocols to prevent producing or propagating
  such non-valley-free BGP paths.
\item ISPs can plan for better future contractual agreements.
\end{itemize}

\vspace{0.5em}
To answer these questions, we need to:
\vspace{0.5em}
\begin{itemize}[nolistsep]
\item develop techniques to identify those BGP paths that are not
  valley-free from the collected BGP data. 
\item classify BGP paths that are not valley-free, that is, is it on
  a customer-to-provider edge or peer-to-peer edge?
\item develop methodology to evaluate the impact of these violations.
\item develop methods to prevent such violations.
\end{itemize}

\vspace{0.5em}
Another thing worth mentioning is that we should be able to use the
inferred AS relationships proposed by the paper to verify our results.

\end{document}
