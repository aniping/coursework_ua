%%%%%%%%%%%%%%%%%%%%%%%%%%%%%%%%%%%%
% LaTeX template for reading report
% Author: Shuo Yang
%%%%%%%%%%%%%%%%%%%%%%%%%%%%%%%%%%%%

\documentclass[11pt]{article}
\usepackage{amsmath,amssymb,epsfig,graphics,hyperref,amsthm,mathtools,enumitem}
\DeclarePairedDelimiter\ceil{\lceil}{\rceil}
\DeclarePairedDelimiter\floor{\lfloor}{\rfloor}

\hypersetup{colorlinks=true}

\setlength{\textwidth}{7in}
\setlength{\topmargin}{-0.575in}
\setlength{\textheight}{9.25in}
\setlength{\oddsidemargin}{-.25in}
\setlength{\evensidemargin}{-.25in}

\reversemarginpar
\setlength{\marginparsep}{-15mm}

\newcommand{\rmv}[1]{}
\newcommand{\bemph}[1]{{\bfseries\itshape#1}}
\newcommand{\N}{\mathbb{N}}
\newcommand{\Z}{\mathbb{Z}}
\newcommand{\imply}{\to}
\newcommand{\bic}{\leftrightarrow}

% Some user defined strings for the homework assignment
%
\def\CourseCode{CS525}
\def\ReportNo{7}
\def\Category{Reading Report}
\def\PaperTitle{A study of prefix hijacking and interception in the
  Internet}
\def\Author{Shuo Yang}

\begin{document}

\noindent

\CourseCode \hfill \Category

\begin{center}
Reading Report \#\ReportNo\\
Paper: \PaperTitle\\
Student: \Author\\
\end{center}

% A horizontal split line
\hrule\smallskip
\vspace{1.5em}

In the paper the authors briefly mentioned two other possibilities an
AS could use for hijacking traffic to a prefix: 1) advertise a more
specific prefix (sub-prefix); 2) advertise a less specific
prefix (super-prefix). Then they stopped right here saying that ``the
impact of such advertisements can be trivially predicated, we don't
study them here''. However, it seems to me not convincing. Although
the intermediate consequence of these kind of hijackings are obvious,
their deep impact to the Internet routing system may be far from being
trivially predicated. In addition, an AS can also advertise an unused
prefix or unallocated prefix. These types of hijacking seem more
suspicious (could be done intentionally) than the regular one
(advertisement of the same prefix as the one being advertised by the
owner) the paper focused on.

\vspace{1em}
It is a pity the paper does not cover different types of prefix
hijacking. In order to get a big picture of the degree of prefix
hijacking and the impact it has imposed on the Internet, it is very
important for us to understand characteristics of different types of
prefix hijacking in deep. We should be interested in knowing: 

\begin{enumerate}
\item What is the frequency distribution of different types of prefix
  hijacking? Which one is more frequently happened and which is less
  frequently happened?
\item What is the fraction of traffic that can be hijacked for
  different types of prefix hijacking?
\item What are the possible root causes for different types of prefix
  hijacking? Are they different by type? For example,
  mis-configuration might cause more exact-prefix hijacking while
  malicious attack might cause more sub-prefix hijacking.
\item Does the type of prefix hijacking correlate with the level of
  tier ASes belong to?
\item How does the BGP routing policy influence the different kind of
  prefix hijacking? For example, the longest match rule causes all
  traffic be routed to sub-prefix hijacking ASes. BGP routing policy
  might have more subtle influences on all kinds of prefix hijacking.
\end{enumerate}

Trying to answer these questions are important because:
\begin{enumerate}
\item It helps us to better understand the global impact of prefix
  hijacking to the entire Internet routing system.
\item It helps us to better detect or prevent prefix hijacking. If we
  know characteristics of different types of prefix hijacking, we can
  devise and deploy different detection or prevent strategies. For
  example, we may adopt more aggressive strategy on sub-prefix
  hijacking, and employ less aggressive strategy on super-prefix
  hijacking due to different level of impact each has.
\end{enumerate}

\end{document}
