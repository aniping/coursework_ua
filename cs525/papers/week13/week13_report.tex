%%%%%%%%%%%%%%%%%%%%%%%%%%%%%%%%%%%%
% LaTeX template for reading report
% Author: Shuo Yang
%%%%%%%%%%%%%%%%%%%%%%%%%%%%%%%%%%%%

\documentclass[11pt]{article}
\usepackage{amsmath,amssymb,epsfig,graphics,hyperref,amsthm,mathtools,enumitem}
\DeclarePairedDelimiter\ceil{\lceil}{\rceil}
\DeclarePairedDelimiter\floor{\lfloor}{\rfloor}

\hypersetup{colorlinks=true}

\setlength{\textwidth}{7in}
\setlength{\topmargin}{-0.575in}
\setlength{\textheight}{9.25in}
\setlength{\oddsidemargin}{-.25in}
\setlength{\evensidemargin}{-.25in}

\reversemarginpar
\setlength{\marginparsep}{-15mm}

\newcommand{\rmv}[1]{}
\newcommand{\bemph}[1]{{\bfseries\itshape#1}}
\newcommand{\N}{\mathbb{N}}
\newcommand{\Z}{\mathbb{Z}}
\newcommand{\imply}{\to}
\newcommand{\bic}{\leftrightarrow}

% Some user defined strings for the homework assignment
%
\def\CourseCode{CS525}
\def\ReportNo{13}
\def\Category{Reading Report}
\def\PaperTitle{A Case for End System Multicast}
\def\Author{Shuo Yang}

\begin{document}

\noindent

\CourseCode \hfill \Category

\begin{center}
Reading Report \#\ReportNo\\
Paper: \PaperTitle\\
Student: \Author\\
\end{center}

% A horizontal split line
\hrule\smallskip
\vspace{1.5em}

Narada protocol is suitable only for multicast application with a
small group size (a few hundred nodes), for example, video
conferencing. Beyond all the well-known 
benefits of implementing multicast functionality on a overlay network,
the major problem of Narada is scalability. This is because:

\begin{enumerate}
\item Each node in the group needs to carry out the duplicating and
  forwarding operations, manage group membership, as well as
  performing application specific functionalities. The first two are
  usually implemented by multicast router in IP multicast.
\item Each node in the overlay network has to maintain a list of all
  the nodes that are currently in the network. This is too much
  overhead for a end host when the size of the overlay network scales
  up.
\item The size of routing table for each node is linear to the size of
  the group. Therefore the state maintained by a node is in the order
  of $O(N)$ where $N$ is the size of group.
\item Since each node manages both data plane and control plane, the
  control overhead is big because each control message contains the
  state of all nodes in the group, thus the control overhead is in the
  order of $O(N^2)$ where $N$ is the size of group.
\end{enumerate}

Another potential problem of Narada is that it takes a long time to
converge and the overlay network tends to be instable because
simulation results show that it is difficult to form a stable mesh,
even after a very long time. 

\vspace{1em}
I think in order to make the idea of end-system-multicast scale to
large size, we need to let the network layer and end system cooperate,
each sharing some workload. For example, one of the most challenging
tasks in overlay networks is the management of node dynamics, that is,
let nodes join and leave the group at their will. In traditional IP
multicast network, node dynamics is not a major concern as the
network infrastructure is considered to be stable and only the leaf
nodes join and leave the network. But for overlay network, this is not
the case any more, node dynamics is completely managed by end system.
This may affect application performance, especially, those real time
network application may suffer from managing node dynamics. 

\vspace{1em}
The combination of the two different multicast schemes can be
adopted for general-purpose scalable multicast network. Though we need 
to carefully evaluate the division of functionalities and try to put
as less function to network as possible. But still, network itself
needs to play a role here.


\vspace{1em}


\end{document}
