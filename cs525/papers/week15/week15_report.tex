%%%%%%%%%%%%%%%%%%%%%%%%%%%%%%%%%%%%
% LaTeX template for reading report
% Author: Shuo Yang
%%%%%%%%%%%%%%%%%%%%%%%%%%%%%%%%%%%%

\documentclass[11pt]{article}
\usepackage{amsmath,amssymb,epsfig,graphics,hyperref,amsthm,mathtools,enumitem}
\DeclarePairedDelimiter\ceil{\lceil}{\rceil}
\DeclarePairedDelimiter\floor{\lfloor}{\rfloor}

\hypersetup{colorlinks=true}

\setlength{\textwidth}{7in}
\setlength{\topmargin}{-0.575in}
\setlength{\textheight}{9.25in}
\setlength{\oddsidemargin}{-.25in}
\setlength{\evensidemargin}{-.25in}

\reversemarginpar
\setlength{\marginparsep}{-15mm}

\newcommand{\rmv}[1]{}
\newcommand{\bemph}[1]{{\bfseries\itshape#1}}
\newcommand{\N}{\mathbb{N}}
\newcommand{\Z}{\mathbb{Z}}
\newcommand{\imply}{\to}
\newcommand{\bic}{\leftrightarrow}

% Some user defined strings for the homework assignment
%
\def\CourseCode{CS525}
\def\ReportNo{15}
\def\Category{Reading Report}
\def\PaperTitle{Networking Named Content}
\def\Author{Shuo Yang}

\begin{document}

\noindent

\CourseCode \hfill \Category

\begin{center}
Reading Report \#\ReportNo\\
Paper: \PaperTitle\\
Student: \Author\\
\end{center}

% A horizontal split line
\hrule\smallskip
\vspace{1.5em}
The paper proposed an fundamentally different approach to the Internet
design than the current Internet architecture. It intended to shift
from the current IP-centric networking to a future content-centric
networking where it views data, but not hosts, as the network
primitive. Such a radically different idea is very interesting and
intriguing, but I want to raise some concerns to this named-content
networking approach.

\begin{enumerate}
\item \emph{Performance concern}\\ Named content networking uses a PKI
  (public key infrastructure) 
  to bind a name to a public key. Every content object needs to be
  signed for privacy concern. For the current Internet, we already have a large amount of
  web servers who sign data or use SSL for privacy reasons. And this
  is just for end-to-end connectivity. If we shift the architecture to
  named content where each content object is unique, it will cause
  even more security signature. How do we 
  manage public key globally? Does every content object need to be
  signed? Should each router forwarding a content object verify its integrity?
  These questions are critically for named content networking
  in terms of performance. Implementing security at the content level is good
  for privacy but can impact performance negatively.
\item \emph{Inter-domain routing}\\
  The paper did not say much about inter-domain routing within the named
  content networking. It illustrated the problem of bottom-up
  deployment and proposed to use the current BGP inter-domain
  routing. But we know that BGP is problematic by its own and
  using an old inter-domain routing for a completely new Internet
  architecture does not sound that convincing.
\item \emph{Applications}\\
  The current Internet is sender driven, which means communications
  are centralized around content providers. So even though many users
  are interested in the same piece of content (for example, a popular
  Youtube video) and they do not care where they get it, each of them
  still has to establish a connection with Youtube server to get that
  content. Such observation and the fact that current Internet is more
  and more content-centric are the underlying motivation of the named
  content networking, trying to turn the Internet into a
  receiver-driven network. However, Internet is rich of a variety of
  applications, still there are many applications favor sender driven
  model. For example, email service and instance messaging. Such
  applications still emphasize host to host connectivity, which the
  current TCP/IP seems to better fit them than the named content
  networking. So how to enable the named content networking to satisfy
  the needs of different network applications (not only for content
  delivery) remains a challenging.
\end{enumerate}

\vspace{1em}

\end{document}
