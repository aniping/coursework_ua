%%%%%%%%%%%%%%%%%%%%%%%%%%%%%%%%%%%%%%%%%%%%%%%%%%%
% LaTeX template for grading report of CS460-Prog1
% Author: Shuo Yang
%%%%%%%%%%%%%%%%%%%%%%%%%%%%%%%%%%%%%%%%%%%%%%%%%%%

\documentclass[10pt]{article}
\usepackage{amsmath,amssymb,epsfig,graphics,hyperref,amsthm,mathtools,enumitem,framed}
\DeclarePairedDelimiter\ceil{\lceil}{\rceil}
\DeclarePairedDelimiter\floor{\lfloor}{\rfloor}

\hypersetup{colorlinks=true}

\setlength{\textwidth}{7in}
\setlength{\topmargin}{-0.575in}
\setlength{\textheight}{9.25in}
\setlength{\oddsidemargin}{-.25in}
\setlength{\evensidemargin}{-.25in}

\reversemarginpar
\setlength{\marginparsep}{-15mm}

\newcommand{\rmv}[1]{}
\newcommand{\bemph}[1]{{\bfseries\itshape#1}}
\newcommand{\N}{\mathbb{N}}
\newcommand{\Z}{\mathbb{Z}}
\newcommand{\imply}{\to}
\newcommand{\bic}{\leftrightarrow}

% Some user defined strings for the homework assignment
%
\def\CourseCode{CS460-Spring16}
\def\Category{Grading Criteria}
\def\ProgNo{1}
\def\Grader{Shuo Yang}

\begin{document}

\noindent

\CourseCode \hfill \Category

\begin{center}
Grading Criteria: Prog \#\ProgNo\\
Grader: \Grader\\
\end{center}

% A horizontal split line
\hrule\smallskip

\section{Documentation \& Coding style (30 points)}

\subsection{Documentation (20 points)}

\begin{tabular}{ | l | l | l | }
  \hline
  & Grading criteria \\ \hline
  External Documentation (5) & Refer to Dr. McCann's Programming Style
  Requirements\\ \hline
  Class Documentation (5) & Refer to Dr. McCann's Programming Style
  Requirements \\ \hline
  Internal Documentation (10) & Refer to Dr. McCann's Programming Style
  Requirements \\ \hline
\end{tabular}

\subsection{Coding style (10 points)}

\begin{tabular}{ | l | l | l | }
  \hline
  & Grading criteria \\ \hline
  Modularity (5) & Refer to Dr. McCann's Programming Style
  Requirements \\ \hline
  Naming (5) & Refer to Dr. McCann's Programming Style
  Requirements \\ \hline
\end{tabular}

\section{Implementation (70 points)}

\begin{tabular}{ | l | l | l | }
  \hline
  & Grading criteria \\ \hline
  Binary file (10) & created successfully \\ \hline
  First 5 and last 5 records (15) & correctly printed out with the
  format given in the spec \\ \hline
  Number of records (5) & print out the correct number \\ \hline
  user interface (5) & wait for user to enter the search key and
  continuously prompt user for input \\ \hline
  Interpolation search (35) & implemented correctly and well functioned \\ \hline
\end{tabular}

\vspace{3em}
Correct output:\\

\begin{framed}
  \noindent
  First 5 records:\\
  26300005,23,5,2.896,-47,33,11.56,356.26205,-47.55321,80,78,10.8,-16.5,7.3,7.2,18.85,0.0,0.0,24,19,1,1,0,,K,
  26300018,23,5,16.318,-46,54,28.92,356.318,-46.90803,192,189,21.5,16.8,7.4,7.2,17.23,0.0,0.74,26,19,1,1,1,,G,
  26300026,23,5,0.094,-46,22,22.61,356.2504,-46.372948,32,91,-29.9,-5.4,5.8,6.5,14.76,0.0,1.05,14,7,1,2,1,,K,
  26300038,23,5,48.135,-45,29,4.81,356.45056,-45.48467,55,51,5.4,-9.5,5.8,5.6,18.3,0.0,0.0,16,25,1,1,0,,K,
  26300048,23,5,49.44,-44,58,52.68,356.456,-44.9813,102,100,-29.8,-5.7,8.0,7.8,17.69,0.0,0.0,6,34,1,1,0,,G,\\
  Last 5 records:\\
  59803186,23,4,38.975,-22,9,28.47,358.66238,-22.157907,86,86,-9.5,0.8,6.5,6.5,17.96,0.0,0.0,10,22,1,1,0,,G,
  59803221,23,5,57.503,-26,30,58.35,358.9896,-26.516207,118,118,-65.6,24.1,8.2,8.2,18.59,0.0,0.0,2,24,1,1,0,,G,
  59803228,23,6,15.396,-26,15,43.92,359.06415,-26.2622,510,141,34.3,32.0,11.5,11.5,18.52,0.0,0.47,30,61,1,1,1,,G,
  59803230,23,5,50.702,-26,7,5.9,358.96127,-26.118305,36,54,-3.0,-2.0,3.2,3.1,15.95,0.0,0.92,12,8,1,1,1,,,S
  59803288,23,5,47.114,-22,34,50.09,358.94632,-22.580582,328,561,-22.7,15.9,9.7,9.7,18.44,0.0,1.54,23,23,1,1,1,,G, 
  \noindent
  Total number of records: 321608
\end{framed}


\end{document}
