\documentclass{article}

\usepackage{fancyvrb}
\usepackage{graphicx}

\setlength{\topmargin}{-0.75in}  % suddenly too close to page top!
\setlength{\textheight}{9.5in}	% length of text block on page (old = 8.25)
\setlength{\textwidth}{6.625in}	% width of text block on page
\setlength{\oddsidemargin}{0.0in}	% left edge to left margin - 1in (375)
\setlength{\evensidemargin}{0.0in}	% same as odd, but for even pages

\setlength{\parindent}{0.0cm}	% Don't indent the paragraphs
\setlength{\parskip}{0.4cm}	% distance between paragraphs

\begin{document}
\pagestyle{empty}			% empty=no page#s; plain=page#s

\begin{center}
CSc 460 --- Database Design \\
Spring 2016 (McCann)

http://www.cs.arizona.edu/classes/cs460/spring16

\fontfamily{phv}\fontseries{b}\selectfont   % Selects the Postscript Helvetica 
                                            % (phv) font in (b)old.
{\large Program \#4:  Database-driven Web Application}
\normalfont

{\it Due Date:  May 6$^{\, th}$, 2016, at the beginning of class}

Designed by {\it Shuo Yang}

\end{center}


\textbf{Overview:}
In this assignment, you will build a database-driven web information
management system from ground up. We will give you an application
domain to work on, your goal is to design the underlying database and
define the application functionalities you will provide with the
database, and finally implement this application as a web-based system.


\textbf{Assignment:}
In this assignment you are to implement a three-tier client-server
architecture.

\begin{enumerate}
\item \textbf{Database Back-End}, which runs the Oracle DBMS
  on \texttt{aloe.cs.arizona.edu}. Your job is to design the database relational
  schema, create tables and populate your tables with some initial
  data. We are requiring that you create an ER diagram, analyze the
  FDs of each table and apply table normalization techniques to your
  schema to justify that your schema satisfies 3NF, and if possible,
  BCNF.

\item \textbf{The business logic and data processing layer}, which is
  the middle tier that runs on an application server, which, in this
  assignment, will be \texttt{lectura.cs.arizona.edu} running the Tomcat
  web server. This layer sits in the middle, receives requests from
  client application and generates response back to client
  application. The response generation may involve accessing the
  back-end database you have created. Though there are many
  server-side techniques available for use, in this assignment we are
  requiring that you use Java and JavaServer Pages (JSP).

\item \textbf{Web Front-End}, which is the client user-interface. You
  need to design webpages appropriately to handle all the required
  functionalities. Your client application can run in any machine
  within the CS department with a web browser installed.
\end{enumerate}


%If you need to force a page break...
%\vfill
%\centerline{(Continued...)}
%\newpage


\textbf{Application Domain:} 
The problem description for this project is as follows:

A University department needs to keep current information about
its students, their academic advisors, the student clubs to which they
belong, the advisors of the clubs, and the specific activities that
those clubs sponsor (such as guest speakers, career nights, etc.).

Each student is assigned to one academic advisor, but of course each
advisor advises many students.  Academic advisors may be faculty members,
but may instead be staff members.

Students can belong to as many clubs as they wish,
and clubs can sponsor any number of activities, but each club must have at
least one member to exist.  Club activities are sponsored by only one club,
but the department happily allows multiple club activities per day.
%Club advisors, like academic advisors, may be either faculty or staff.

The department has both undergraduate and graduate students.  Graduate students
may be club members, but they can also be club advisors, as can faculty
members or staff members.  Graduate students also each have an academic
advisor, and may optionally have a thesis advisor, who must be a faculty
member.


\textbf{Required functionalities:}
\begin{enumerate}
\item \textbf{Record insertion}.
Your application should support inserting a new data record via web
interface.

\item \textbf{Record deletion}.
Your application should support deleting an existing data record via
web interface.

\item \textbf{Record update}.
Your application should support updating an existing data record via
web interface.

\item \textbf{Record query}.
Your application should support querying your database via the web
interface for the problem description given above.
You are required to implement five different queries, each of your own
design, but with the following restrictions: ONE must be constructed
using at least one piece of information gathered from the user. ONE
must involve at least two relations. There should no trivial queries (
for example, simply selecting everything from a table), you queries need
to be able to answer questions that real users are interested in.
\end{enumerate}

More specifically, you should support record insertion and deletion
for ALL tables you designed, and support record update for ONLY one
table of your own choice. For each table you created, you need to
populate a reasonable number of rows in order to test your
queries. (50-100 rows seem reasonable, while 5-10 rows may be not)

\textbf{Work in Groups:}
In industry, projects are usually the work of
multiple developers, since it involves several different components.
Good communication is a vital key to the success of the project. This
homework provides such an opportunity for teamwork. Therefore, it is
required that groups of 2-4 members should be formed.

You need to come up with a reasonable workload distribution
scheme. More importantly, you need to come up with a well-formed
design at the beginning. This will save a lot of extra conflicts and
debugging efforts in the actual implementation.

\emph{Note}: each team should email the TA (Shuo Yang) at
\texttt{shuoyang@email.arizona.edu} the members
of the team no later than April 25 (Monday).


\textbf{Hand In:}
You are required to submit a \texttt{.tar} file of your well-documented
application program file(s) via turnin to the folder
\texttt{cs460p4}. The tar file should contain the following exactly: 
\begin{enumerate}
\item A directory called ``ROOT'', which contains all the source code
  for the application. The structure of it should follow exactly as
  what you will see in the simple demo application (see below). 
\item A directory called ``doc'', which contains one PDF document
  including the sections in the following order:
  \begin{enumerate}
  \item
    \emph{Conceptual database design}: ER diagram along with your design
    rationale and any necessary high-level text description of
    the data model (e.g., constraints or anything not able to show in
    the ER diagram but is necessary to help people understand your
    database design).
  \item
    \emph{Logical database design}: converting an ER schema into a
    relational database schema. Show the schemas of the tables resulted
    in this step.
  \item
    \emph{Normalization analysis}: show FDs of all your tables and justify
    why your design adheres to 3NF.
  \item
    \emph{Query description}: describe your five queries, what
    questions they are answering?
  \end{enumerate}

\item A \texttt{ReadMe.txt} describing how TA can operate your
  website to see the required functionalities, and work load
  distribution among team members (that is, who is responsible for
  what?). If there are any required functionalities you that you have
  failed to implement, list them.
\end{enumerate}
Each team should schedule a time slot (10 minutes) to meet with the TA
(Shuo Yang) and demonstrate your system. We will let you know how to
sign up later.


\textbf{A Simple Demo Application:}
To speed up the development, I've put a simple demo application under
\texttt{/home/cs460/spring16/2016p4/}, please read the
\texttt{HowTo.txt} within it to see how to run the demo. The demo
contains a simple web page with a button you can click. By clicking the button, it
will retrieve all the records from table \texttt{mccann.employee} and
display the content on another web page.

You should run this demo because 1) it will let you install
the Tomcat web server under your account which is needed for your
application to run; 2) it
will help you get
familiar with the techniques you are going to use for this assignment.

\textbf{Grading:} Total: 100
\begin{enumerate}
\item Coding style: 5
\item Database design: 50
  \begin{itemize}
  \item Conceptual database design: 20
  \item Logical database design: 10
  \item Normalization analysis: 20
  \end{itemize}
\item Implementation: 45
  \begin{itemize}
  \item Record insertion: 10
  \item Record deletion: 10
  \item Record update: 5
  \item Query: 15
  \item web front-end: 5
  \end{itemize}
\end{enumerate}
\emph{Note}: We won't put too much weight on the look of the web
pages. The main point of the assignment is the DB design, the
web side is a nice bonus. You should put your focus on designing web
pages properly (functional) to handle the required functionalities,
don't worry if your web pages don't look ``nice.''

\textbf{Late days:} Late days can be used on this assignment; how many
a team has to use is determined as follows: Team members total their
remaining late days, divide by the number of members in the team
(integer division), and that's the number of late days the team has
available, to a max of 'two' days.  ('Two?'  With 'quotes?'  We need
to get grading done soon after the final, so we can't let any programs
be submitted after the start of the final.  As the final is at 1:00
p.m. Friday and the due date is at 2:00 p.m. Wednesday, you have 47
hours, not 48.  So, 'two' late days max.)

For example, a team whose three members have 1, 1, and 3 late days
remaining have (1+1+3) / 3 = 1 late day to use to get their project
materials submitted.


\end{document}

