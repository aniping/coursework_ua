%%%%%%%%%%%%%%%%%%%%%%%%%%%%%%%%%%%%%%%%%%%%%%%%%%%%%%%%%%%%%%%%%%%%%%%%%%%
% This is LaTeX file for Homework Assignment 1 of CS460
% Author: Shuo Yang
%%%%%%%%%%%%%%%%%%%%%%%%%%%%%%%%%%%%%%%%%%%%%%%%%%%%%%%%%%%%%%%%%%%%%%%%%%%

\documentclass[10pt]{article}
\usepackage{amsmath,amssymb,epsfig,graphics,hyperref,amsthm,mathtools}
\DeclarePairedDelimiter\ceil{\lceil}{\rceil}
\DeclarePairedDelimiter\floor{\lfloor}{\rfloor}

\hypersetup{colorlinks=true}

\setlength{\textwidth}{7in}
\setlength{\topmargin}{-0.575in}
\setlength{\textheight}{9.25in}
\setlength{\oddsidemargin}{-.25in}
\setlength{\evensidemargin}{-.25in}

\reversemarginpar
\setlength{\marginparsep}{-15mm}

\newcommand{\rmv}[1]{}
\newcommand{\bemph}[1]{{\bfseries\itshape#1}}
\newcommand{\N}{\mathbb{N}}
\newcommand{\Z}{\mathbb{Z}}
\newcommand{\imply}{\to}
\newcommand{\bic}{\leftrightarrow}

% Some user defined strings for the homework assignment
%
\def\CourseCode{CS460}
\def\AssignmentNo{4}
\def\DateHandedOut{Spring, 2016}
\def\Author{Shuo Yang}

\begin{document}

\noindent

\CourseCode \hfill \DateHandedOut

\begin{center}
Homework \#\AssignmentNo\\
TA: \Author\\
\end{center}

% A horizontal split line
\hrule\smallskip

% Enumerate through all questions.
\begin{enumerate}
\item 5.12 (d)\\
  RA:
  \begin{align*}
    \pi_{price, type}(Room \bowtie_{Room.hotelNo=Hotel.hotelNo}
    (\sigma_{hotelName=``Grosvenor Hotel''}(Hotel)))
  \end{align*}

  TRC:
  \begin{align*}
    \{R.price, R.type &| Room(R) \land (\exists H)(Hotel(H) \land
    (R.hotelNo=H.hotelNo)\\ &\land (H.hotelName = ``Grosvenor Hotel'')) \}
  \end{align*}

  DRC:
  \begin{align*}
    \{price, type &| (\exists
    roomNo,hotelNo,hotelNo1,hotelName,city)\\
    &(Room(roomNo, hotelNo, type, price) \land Hotel(hotelNo1,
    hotelName, city)\\
    &\land  (hotelNo = hotelNo1) \land (hotelName = ``Grosvenor Hotel''))\}
  \end{align*}

\item 5.26\\
  RA:
  \begin{align*}
    \pi_{title}(\sigma_{year=2012}(Book))
  \end{align*}

  TRC:
  \begin{align*}
    \{B.title | Book(B) \land B.year=2012\}
  \end{align*}

  DRC:
  \begin{align*}
    \{title | (\exists ISBN, edition, year) (Book(ISBN, title,
    edition, year) \land year=2012)\}
  \end{align*}
  
\item 5.28\\
  RA:
  \begin{align*}
    \pi_{copyNo}(\sigma_{title=``Load of the Rings''}(Book)
    \bowtie_{ISBN} (\sigma_{available='Y'}(BookCopy)))
  \end{align*}

  TRC:
  \begin{align*}
    \{BC.copyNo &| BookCopy(BC) \land (\exists B)(Book(B) \land (B.ISBN
    = BC.ISBN)\\ &\land (BC.available='Y') \land (B.title=``Lord of the Rings''))\}
  \end{align*}

  DRC:
  \begin{align*}
    \{copyNo &| (\exists ISBN, title, edition, year, ISBN, available)
    (Book(ISBN, title, edition, year)\\ &\land BookCopy(copyNo, ISBN,
    available) \land available='Y' \land title=``Lord of the Rings'')\}
  \end{align*}

\item 6.4\\
  An aggregate function can be used only in the SELECT list and in the
  HAVING clause.

  Apart from COUNT(*), each function eliminates nulls first and
  operates only on the remaining non-null values. COUNT(*) counts all
  the rows of a table, regardless of whether nulls or duplicate values
  occur.

\item 6.10\\
  SELECT * FROM Room WHERE price $<$ 40 AND type IN ('D', 'F')
  ORDER BY price;\\

  (Note, ASC is the default setting for ORDER BY).

\item 6.14\\
  SELECT SUM(price) FROM Room WHERE type = 'D';

\item 6.20\\
  SELECT * FROM Room r\\
  WHERE roomNo NOT IN\\
  (SELECT roomNo FROM Booking b, Hotel h\\
  WHERE (dateFrom $<=$ CURRENT\_DATE AND\\
  dateTo $>=$ CURRENT\_DATE) AND\\
  b.hotelNo = h.hotelNo AND hotelName = ``Grosvenor Hotel'');

\item 6.25\\
  SELECT MAX(X)\\
  FROM ( SELECT type, COUNT(type) AS X\\
  FROM Booking b, Hotel h, Room r\\
  WHERE r.roomNo = b.roomNo AND b.hotelNo = h.hotelNo AND\\
  city = ``London'' GROUP BY type);

\item 7.7\\
  Materialized view is a temporary table that is stored in the
  database to represent a view, which is maintained as the base
  table(s) are updated.\\

  Advantages:\\
  - may be faster than trying to perform view resolution.\\
  - may also be useful for integrity checking and query optimization.

\item 7.11 (b,d,f)\\
  CREATE DOMAIN RoomPrice AS DECIMAL(5, 2)\\
  CHECK(VALUE BETWEEN 10 AND 100);\\

  CREATE TABLE Room(\\
  \-\hspace{2em}roomNo integer NOT NULL,\\
  \-\hspace{2em}hotelNo integer NOT NULL,\\
  \-\hspace{2em}type char(1) NOT NULL,\\
  \-\hspace{2em}price RoomPrice NOT NULL,\\
  \-\hspace{2em}PRIMARY KEY (roomNo, hotelNo),\\
  \-\hspace{2em}FOREIGN KEY (hotelNo) REFERENCES Hotel\\
  \-\hspace{2em}ON DELETE CASCADE ON UPDATE CASCADE\\
  );\\

  CREATE TABLE Booking(\\
  \-\hspace{2em}hotelNo integer NOT NULL,\\
  \-\hspace{2em}guestNo integer NOT NULL,\\
  \-\hspace{2em}dateFrom date NOT NULL,\\
  \-\hspace{2em}dateTo date NULL,\\
  \-\hspace{2em}roomNo integer NOT NULL,\\
  \-\hspace{2em}PRIMARY KEY (hotelNo, guestNo, dateFrom),\\
  \-\hspace{2em}FOREIGN KEY (hotelNo) REFERENCES Hotel\\
  \-\hspace{2em}ON DELETE CASCADE ON UPDATE CASCADE,\\
  \-\hspace{2em}FOREIGN KEY (guestNo) REFERENCES Guest\\
  \-\hspace{2em}ON DELETE NO ACTION ON UPDATE CASCADE,\\
  \-\hspace{2em}FOREIGN KEY (roomNo) REFERENCES Room\\
  \-\hspace{2em}ON DELETE NO ACTION ON UPDATE CASCADE,\\
  \-\hspace{2em}CONSTRAINT GuestBooked\\
  \-\hspace{2em}CHECK (NOT EXISTS (SELECT *\\
  \-\hspace{13.5em}FROM Booking b\\
  \-\hspace{13.5em}WHERE b.dateTo $>$ Booking.dateFrom AND\\
  \-\hspace{13.5em}b.dateFrom $<$ Booking.dateTo AND\\
  \-\hspace{13.5em}b.guestNo = Booking.guestNo)));

\item 7.14\\
  CREATE VIEW BookingOutToday AS\\
  \-\hspace{2em}SELECT g.guestNo, g.guestName, g.guestAddress,
  r.price*(b.dateTo-b.dateFrom)\\
  \-\hspace{2em}FROM Guest g, Booking b, Hotel h, Room r\\
  \-\hspace{2em}WHERE g.guestNo = b.guestNo AND r.roomNo = b.roomNo AND\\
  \-\hspace{2em}b.hotelNo = h.hotelNo AND h.hotelName = ``Grosvenor Hotel'' AND\\
  \-\hspace{2em}b.dateTo = CURRENT\_DATE;

\item 8.5\\
  PL/SQL uses \textbf{cursors} to allow the rows of a query result to be
  accessed one at a time. In effect, the cursor acts as a pointer to a
  particular row of the query result. The cursor can be advanced by 1
  to access the next row. A cursor must be declared and opened before
  it can be used, and it must be closed to deactivate it after it is
  no longer required. Once the cursor has been opened, the rows of the
  query result can be retrieved one at a time using a FETCH statement,
  as opposed to a SELECT statement.

\item 8.7\\
  The BEFORE keyword indicates that the trigger should be executed
  before an insert is applied. It could be used to prevent a member of
  staff from managing more than 100 properties at the same time.\\

  The AFTER keyword indicates that the trigger should be executed
  after the database table is updated. It could be used to create an
  audit record.

\item 8.8\\
  There are two types of trigger: row-level triggers (FOR EACH ROW)
  that execute for each row of the table that is affected by the
  triggering event, and statement-level triggers (FOR EACH STATEMENT)
  that execute only once even if multiple rows are affected by the
  triggering event. An example of a row-level trigger is shown in
  Example 8.2 where we create an AFTER row-level trigger to keep an
  audit trail of all rows inserted into the Staff table. An example of
  a statement-level trigger to set a new sequence number for an update
  is seen in the middle of Figure 8.5(b).

\item 8.11 (b)\\
  CREATE TRIGGER RoomPrice\\
  BEFORE INSERT ON Room\\
  FOR EACH ROW\\
  WHEN (new.type = 'D')\\
  DECLARE vMaxSingleRoomPrice NUMBER;\\
  BEGIN\\
  \-\hspace{2em}SELECT MAX(price) INTO vMaxSingleRoomPrice\\
  \-\hspace{2em}FROM Room\\
  \-\hspace{2em}WHERE type = ‘S’;\\
  \-\hspace{2em}IF (new.price $<$ vMaxSingleRoomPrice)\\
  \-\hspace{4em}raise\_application\_error(-20000, ``Double room price
  must be higher than higest single room price'' $||$ vMaxSingleRoomPrice);\\
  \-\hspace{2em}END IF;\\
  END;\\

\item 14.4\\
  Functional dependency describes the relationship between attributes
  in a relation. For example, if A and B are attributes of relation R,
  B is functionally dependent on A (denoted A $\imply$ B), if each value of A
  in R is associated with exactly one value of B in R.

  Functional dependency is a property of the meaning or semantics of
  the attributes in a relation. The semantics indicate how the
  attributes relate to one another and specify the functional
  dependencies between attributes. When a functional dependency is
  present, the dependency is specified as a constraint between the attributes.

\item 14.6\\
  Identifying all functional dependencies between a set of attributes
  should be relatively simple if the meaning of each attribute and the
  relationships between the attributes are well understood. This type
  of information may be provided by the enterprise in the form of
  discussions with users and/or appropriate documentation such as the
  users’ requirements specification. However, if the users are
  unavailable for consultation and/or the documentation is incomplete
  then depending on the database application it may be necessary for
  the database designer to use their common sense and/or experience to
  provide the missing information.
  As a contrast to this, consider the situation where functional
  dependencies are to be identified in the absence of appropriate
  information about the meaning of attributes and their
  relationships. In this case, it may be possible to identify
  functional dependencies if sample data is available that is a true
  representation of all possible data values that the database may
  hold.

\item 14.7\\
  A table in unnormalized form contains one or more repeating
  groups. To convert to first normal form (1NF) either remove the
  repeating group to a new relation along with a copy of the original
  key attribute(s), or remove the repeating group by entering
  appropriate data in the empty columns of rows containing the
  repeating data.

\item
  Given $C \imply BDEF$ and $D \imply AG$, by the decomposition rule,
  we know that:
  \begin{align*}
    C \imply B\\
    C \imply D\\
    C \imply E\\
    C \imply F\\
    D \imply A\\
    D \imply G
  \end{align*}

  By the transitivity rule, since $C \imply D$ and $D \imply A,D
  \imply G$, we know that:
  \begin{align*}
    C \imply A\\
    C \imply G
  \end{align*}

  By relexivity rule, we know that $C \imply C$. Thus:
  \begin{align*}
    C^+ = ABCDEFG
  \end{align*}

  By the given set of FDs, no attribute functionally dependent on $A$
  except itself, thus:
  \begin{align*}
    A^+ = A
  \end{align*}

\item 1) Put the FDs in a standard form:
  \begin{align*}
    &W \imply Z\\
    &WY \imply X\\
    &Y \imply Z\\
    &W \imply X \text{  [decomposition rule]}\\
    &W \imply Y \text{  [decomposition rule]}
  \end{align*}
  2) Minimize the left side of each FD:\\
  we can replace $WY \imply X$ with $W \imply X$ since $W \imply Y$:
  \begin{align*}
    &W \imply Z\\
    &W \imply X\\
    &Y \imply Z\\
    &W \imply X\\
    &W \imply Y
  \end{align*}
  3) Delete redundant FDs:\\
  one of $W \imply X$ is redundant. And $W \imply Z$ is redundant
  because it can be inferred by $W \imply Y$ and $Y \imply Z$.

  Thus, the minimal cover of $S$ is:
  \begin{align*}
    &W \imply X\\
    &Y \imply Z\\
    &W \imply Y
  \end{align*}

\end{enumerate}
\end{document}
