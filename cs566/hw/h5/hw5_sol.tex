%%%%%%%%%%%%%%%%%%%%%%%%%%%%%%%%%%%%%%%%%%%%%%%%%%%%%%%%%%%%%%%%%%%%%%%%%%%
% This is LaTeX file for Homework Assignment 3
% Author: Shuo Yang
%%%%%%%%%%%%%%%%%%%%%%%%%%%%%%%%%%%%%%%%%%%%%%%%%%%%%%%%%%%%%%%%%%%%%%%%%%%

\documentclass[11pt]{article}
\usepackage{amsmath,amssymb,epsfig,graphics,hyperref,amsthm,mathtools}
\DeclarePairedDelimiter\ceil{\lceil}{\rceil}
\DeclarePairedDelimiter\floor{\lfloor}{\rfloor}

\hypersetup{colorlinks=true}

\setlength{\textwidth}{7in}
\setlength{\topmargin}{-0.575in}
\setlength{\textheight}{9.25in}
\setlength{\oddsidemargin}{-.25in}
\setlength{\evensidemargin}{-.25in}

\reversemarginpar
\setlength{\marginparsep}{-15mm}

\newcommand{\rmv}[1]{}
\newcommand{\bemph}[1]{{\bfseries\itshape#1}}
\newcommand{\N}{\mathbb{N}}
\newcommand{\Z}{\mathbb{Z}}
\newcommand{\imply}{\to}
\newcommand{\bic}{\leftrightarrow}

% Some user defined strings for the homework assignment
%
\def\CourseCode{CS566}
\def\AssignmentNo{5}
\def\DateHandedOut{Fall, 2015}
\def\Author{Shuo Yang}

\begin{document}

\noindent

\CourseCode \hfill \DateHandedOut

\begin{center}
Homework Assignment \#\AssignmentNo\\
Student: \Author\\
\end{center}

% A horizontal split line
\hrule\smallskip

% Enumerate through all questions.
\begin{enumerate}

\item % Problem 1
  Create a subtraction table for $Z_7$, $(x-y)$ mod 7, similar to
  slide 6.

Table is shown below, the row index represents $x$ and the column
index represents $y$.

\begin{tabular}{ l | l | l | l | l | l | l | l }
  $-$ & 0 & 1 & 2 & 3 & 4 & 5 & 6 \\ \hline
  0 & 0 & 6 & 5 & 4 & 3 & 2 & 1 \\ \hline
  1 & 1 & 0 & 6 & 5 & 4 & 3 & 2 \\ \hline
  2 & 2 & 1 & 0 & 6 & 5 & 4 & 3 \\ \hline
  3 & 3 & 2 & 1 & 0 & 6 & 5 & 4 \\ \hline
  4 & 4 & 3 & 2 & 1 & 0 & 6 & 5 \\ \hline
  5 & 5 & 4 & 3 & 2 & 1 & 0 & 6 \\ \hline
  6 & 6 & 5 & 4 & 3 & 2 & 1 & 0 \\
\end{tabular}

\item Compute $GCD(500, 793)$ using method given in slide 10. Show
  your work.

Let $a=793$ and $b=500$, with Euler's GCD algorithm, we have the
following recursive steps:

\emph{iter-1:} $a=793$, $b=500$, $q=\floor{793/500}=1$\\
\emph{iter-2:} $a=500$, $b=793 \mod 500 = 293$, $q=\floor{500/293}=1$\\
\emph{iter-3:} $a=293$, $b=500 \mod 293 = 207$, $q=\floor{293/207}=1$\\
\emph{iter-4:} $a=207$, $b=293 \mod 207 = 86$, $q=\floor{207/86}=2$\\
\emph{iter-5:} $a=86$, $b=207 \mod 86 = 35$, $q=\floor{86/35}=2$\\
\emph{iter-6:} $a=35$, $b=86 \mod 35 = 16$, $q=\floor{35/16}=2$\\
\emph{iter-7:} $a=16$, $b=35 \mod 16 = 3$, $q=\floor{16/3}=5$\\
\emph{iter-8:} $a=3$, $b=16 \mod 3 = 1$, $q=\floor{3/1}=3$\\
\emph{iter-9:} $a=1$, $b=3 \mod 1 = 0$, return $(1, 1, 0)$\\

Now work backwards to reach to the final answer:

\emph{iter-9:} $a=1$, $b=3 \mod 1 = 0$, return $(1, 1, 0)$\\
\emph{iter-8:} $a=3$, $b=16 \mod 3 = 1$, $q=\floor{3/1}=3$,\\
\-\hspace{3em}$(d,k,l)=(1,1,0)$, $d=1, l=0, k-lq=1$, return (1,0,1)\\ 
\emph{iter-7:} $a=16$, $b=35 \mod 16 = 3$, $q=\floor{16/3}=5$,\\
\-\hspace{3em}$(d,k,l)=(1,0,1)$, $d=1, l=1, k-lq=0-1*5=-5$, return (1,1,-5)\\ 
\emph{iter-6:} $a=35$, $b=86 \mod 35 = 16$, $q=\floor{35/16}=2$,\\
\-\hspace{3em}$(d,k,l)=(1,1,-5)$, $d=1, l=-5, k-lq=1-(-5)*2=11$,
return (1,-5, 11)\\
\emph{iter-5:} $a=86$, $b=207 \mod 86 = 35$, $q=\floor{86/35}=2$,\\
\-\hspace{3em}$(d,k,l)=(1,-5,11)$, $d=1, l=11, k-lq=(-5)-11*2=-27$,
return (1,11,-27)\\
\emph{iter-4:} $a=207$, $b=293 \mod 207 = 86$, $q=\floor{207/86}=2$,\\
\-\hspace{3em}$(d,k,l)=(1,11,-27)$, $d=1, l=-27, k-lq=11-(-27)*2=65$,
return (1,-27,65)\\
\emph{iter-3:} $a=293$, $b=500 \mod 293 = 207$, $q=\floor{293/207}=1$,\\
\-\hspace{3em}$(d,k,l)=(1,-27,65)$, $d=1, l=65, k-lq=-27-65*1=-92$,
return (1,65,-92)\\
\emph{iter-2:} $a=500$, $b=793 \mod 500 = 293$,$q=\floor{500/293}=1$,\\
\-\hspace{3em}$(d,k,l)=(1,65,-92)$, $d=1, l=-92, k-lq=65-(-92)*1=157$,
return (1,-92,157)\\
\emph{iter-1:} $a=793$, $b=500$, $q=\floor{793/500}=1$\\
\-\hspace{3em}$(d,k,l)=(1,-92,157)$, $d=1, l=157, k-lq=-92-157*1=-249$,
return (1,157,-249)\\

Therefore $GCD(500, 793) = (1, -249, 157)$.

\item Compute $GCD(720,999)$ using method given in slide 12. Show your
  work.

$999 = 720 * 1 + 279$\\
$720 = 279 * 2 + 162$\\
$279 = 162 * 1 + 117$\\
$162 = 117 * 1 + 45$\\
$117 = 45 * 2 + 27$\\
$45 = 27 * 1 + 18$\\
$27 = 18 * 1 + \textbf{9}$\\
$18 = 9 * 2 + 0$\\

Therefore, $GCD(720,999) = 9$.

\item Compute $i$ and $j$ such that $GCD(500, 793) = i * 500 + j *
  793$. Show your work.

1: 793 = 500 $*$ 1 + 293\\
2: 500 = 293 $*$ 1 + 207\\
3: 293 = 207 $*$ 1 + 86\\
4: 207 = 86  $*$ 2  + 35\\
5: 86  = 35  $*$ 2  + 16\\
6: 35  = 16  $*$ 2  + 3\\
7: 16  = 3   $*$ 5 + \underline{\textbf{1}}\\
8: 3   = 1   $*$ 3 + 0\\

So $GCD(500, 793) = 1$.

Transform the equtions 1 to 7 as:

1: 793 - 500 $*$ 1 = 293\\
2: 500 - 293 $*$ 1 = 207\\
3: 293 - 207 $*$ 1 = 86\\
4: 207 - 86  $*$ 2  = 35\\
5: 86  - 35  $*$ 2  = 16\\
6: 35  - 16  $*$ 2  = 3\\
7: 16  - 3   $*$ 5 = \underline{\textbf{1}}\\

Now work from 2 to 7:

2: 500 - 293 $*$ 1 = 207\\
substitute 293 with (793 - 500 $*$ 1):\\ 
500 - (793 - 500 $*$ 1) $*$ 1 = 207\\
793 $*$ -1 + 500 $*$ 2 = 207\\

3: 293 - 207 $*$ 1 = 86\\
substitute 207 with (793 $*$ -1 + 500 $*$ 2), and 293 with (793 - 500 $*$ 1):\\
(793 - 500 $*$ 1) - (793 $*$ -1 + 500 $*$ 2) $*$ 1 = 86\\
793 $*$ 2 + 500 $*$ (-3) = 86\\

4: 207 - 86  $*$ 2  = 35\\
substitute 207 with (793 $*$ -1 + 500 $*$ 2), and 86 with (793 $*$ 2 +
500 $*$ (-3)):\\
(793 $*$ -1 + 500 $*$ 2) - (793 $*$ 2 + 500 $*$ (-3)) $*$ 2 = 35\\
793 $*$ -5 + 500 $*$ 8 = 35\\

5: 86  - 35  $*$ 2  = 16\\
substitute 86 with (793 $*$ 2 + 500 $*$ (-3)): and 35 with (793 $*$ -5
+ 500 $*$ 8)\\
(793 $*$ 2 + 500 $*$ (-3)) - (793 $*$ -5 + 500 $*$ 8) $*$ 2 = 16\\
793 $*$ 12 + 500 $*$ -19 = 16\\

6: 35  - 16  $*$ 2  = 3\\
substitute 35 with (793 $*$ -5 + 500 $*$ 8) and 16 with (793 $*$ 12 +
500 $*$ -19):\\
(793 $*$ -5 + 500 $*$ 8) - (793 $*$ 12 + 500 $*$ -19) $*$ 2 = 3\\
793 $*$ -29 + 500 $*$ 46 = 3\\

7: 16  - 3   $*$ 5 = \underline{\textbf{1}}\\
substitute 16 with (793 $*$ 12 + 500 $*$ -19) and 3 with (793 $*$ -29
+ 500 $*$ 46):\\
(793 $*$ 12 + 500 $*$ -19) - (793 $*$ -29 + 500 $*$ 46) $*$ 5 =
\underline{\textbf{1}}\\ 
793 $*$ 157 + 500 $*$ (-249) = \underline{\textbf{1}}\\

Therefore $GCD(500, 793) = 1 = 500 * (-249) + 793 * 157$, $i = -249, j
= 157$.

\item Compute $i$ and $j$ such that $GCD(720, 999) = i * 720 + j *
  999$. Show your work.

1. 999 = 720 $*$ 1 + 279\\
2. 720 = 279 $*$ 2 + 162\\
3. 279 = 162 $*$ 1 + 117\\
4. 162 = 117 $*$ 1 + 45\\
5. 117 = 45 $*$ 2 + 27\\
6. 45 = 27 $*$ 1 + 18\\
7. 27 = 18 $*$ 1 + \underline{\textbf{9}}\\
8. 18 = 9 $*$ 2 + 0\\

So $GCD(720, 999) = 9$.
Transform the equtions 1 to 7 as:

1. 999 - 720 $*$ 1 = 279\\
2. 720 - 279 $*$ 2 = 162\\
3. 279 - 162 $*$ 1 = 117\\
4. 162 - 117 $*$ 1 = 45\\
5. 117 - 45 $*$ 2 = 27\\
6. 45 - 27 $*$ 1 = 18\\
7. 27 - 18 $*$ 1 = \underline{\textbf{9}}\\

Now work from 2 to 7:

2. 720 - 279 $*$ 2 = 162\\
Substitute 279 with (999 - 720 $*$ 1):\\
720 - (999 - 720 $*$ 1) $*$ 2 = 162\\
720 $*$ 3 - 999 $*$ 2 = 162\\

3. 279 - 162 $*$ 1 = 117\\
Substitute 279 with (999 - 720 $*$ 1) and 162 with (720 $*$ 3 - 999
$*$ 2):\\
(999 - 720 $*$ 1) - (720 $*$ 3 - 999$*$ 2) $*$ 1 = 117\\
720 $*$ -4 + 999 $*$ 3 = 117\\

4. 162 - 117 $*$ 1 = 45\\
Substitute 162 with (720 $*$ 3 - 999 $*$ 2) and 117 with (720 $*$ -4 +
999 $*$ 3):\\ 
(720 $*$ 3 - 999 $*$ 2) - (720 $*$ -4 + 999 $*$ 3) $*$ 1 = 45\\
720 $*$ 7 - 999 $*$ 5 = 45\\

5. 117 - 45 $*$ 2 = 27\\
Substitute 117 with (720 $*$ -4 + 999 $*$ 3) and 45 with (720 $*$ 7 -
999 $*$ 5):\\
(720 $*$ -4 + 999 $*$ 3) - (720 $*$ 7 - 999 $*$ 5) $*$ 2 = 27\\
720 $*$ -18 + 999 $*$ 13 = 27\\

6. 45 - 27 $*$ 1 = 18\\
Substitute 45 with (720 $*$ 7 - 999 $*$ 5) and 27 with (720 $*$ -18 +
999 $*$ 13):\\
(720 $*$ 7 - 999 $*$ 5) - (720 $*$ -18 + 999 $*$ 13) $*$ 1 = 18\\
720 $*$ 25 - 999 $*$ 18 = 18\\

7. 27 - 18 $*$ 1 = \underline{\textbf{9}}\\
Substitute 27 with (720 $*$ -18 + 999 $*$ 13) and 18 with (720 $*$ 25
- 999 $*$ 18):\\ 
(720 $*$ -18 + 999 $*$ 13) - (720 $*$ 25 - 999 $*$ 18) $*$ 1 =
\underline{\textbf{9}}\\
720 $*$ (-43) + 999 $*$ 31 = \underline{\textbf{9}}

Therefore $GCD(720, 999) = 9 = 720 * (-43) + 999 * 31$, $i = -43, j
= 31$.

\item Create a modular multiplication table for $Z_{13}$, xy mod 13 and
  highlight inverses.

Inverses are highlighted as \underline{\textbf{1}}.\\

\begin{tabular}{ l | l | l | l | l | l | l | l | l | l | l | l | l | l}
  $\times$ & 0 & 1 & 2 & 3 & 4 & 5 & 6 & 7 & 8 & 9 & 10 & 11 &
  12\\ \hline
  0 & 0 & 0 & 0 & 0 & 0 & 0 & 0 & 0 & 0 & 0 & 0 & 0 & 0\\ \hline
  1 & 0 & \underline{\textbf{1}} & 2 & 3 & 4 & 5 & 6 & 7 & 8 & 9 & 10
  & 11 & 12\\ \hline
  2 & 0 & 2 & 4 & 6 & 8 & 10 & 12 & \underline{\textbf{1}} & 3 & 5 & 7
  & 9 & 11\\ \hline
  3 & 0 & 3 & 6 & 9 & 12 & 2 & 5 & 8 & 11 & \underline{\textbf{1}} & 4 & 7 & 10\\ \hline
  4 & 0 & 4 & 8 & 12 & 3 & 7 & 11 & 2 & 6 & 10 & \underline{\textbf{1}} & 5 & 9\\ \hline
  5 & 0 & 5 & 10 & 2 & 7 & 12 & 4 & 9 & \underline{\textbf{1}} & 6 & 11 & 3 & 8\\ \hline
  6 & 0 & 6 & 12 & 5 & 11 & 4 & 10 & 3 & 9 & 2 & 8 & \underline{\textbf{1}} & 7\\ \hline
  7 & 0 & 7 & \underline{\textbf{1}} & 8 & 2 & 9 & 3 & 10 & 4 & 11 & 5 & 12 & 6\\ \hline
  8 & 0 & 8 & 3 & 11 & 6 & \underline{\textbf{1}} & 9 & 4 & 12 & 7 & 2 & 10 & 5\\ \hline
  9 & 0 & 9 & 5 & \underline{\textbf{1}} & 10 & 6 & 2 & 11 & 7 & 3 & 12 & 8 & 4\\ \hline
  10 & 0 & 10 & 7 & 4 & \underline{\textbf{1}} & 11 & 8 & 5 & 2 & 12 &
  9 & 6 & 3\\ \hline
  11 & 0 & 11 & 9 & 7 & 5 & 3 & \underline{\textbf{1}} & 12 & 10 & 8 & 6 & 4 & 2\\ \hline
  12 & 0 & 12 & 11 & 10 & 9 & 8 & 7 & 6 & 5 & 4 & 3 & 2 & \underline{\textbf{1}}
\end{tabular}


\item Create a modular exponentiation table for $Z_{11}$, $x^y$ mod 11.\\\\\\\\\\\\

\begin{tabular}{ l | l | l | l | l | l | l | l | l | l | l }
   & y & y & y & y & y & y & y & y & y & y\\ \hline
  exp & 1 & 2 & 3 & 4 & 5 & 6 & 7 & 8 & 9 & 10\\ \hline
  $1^y$ & 1 & 1 & 1 & 1 & 1 & 1 & 1 & 1 & 1 & 1\\ \hline
  $2^y$ & 2 & 4 & 8 & 5 & 10 & 9 & 7 & 3 & 6 & 1\\ \hline
  $3^y$ & 3 & 9 & 5 & 4 & 1 & 3 & 9 & 5 & 4 & 1\\ \hline
  $4^y$ & 4 & 5 & 9 & 3 & 1 & 4 & 5 & 9 & 3 & 1\\ \hline
  $5^y$ & 5 & 3 & 4 & 9 & 1 & 5 & 3 & 4 & 9 & 1\\ \hline
  $6^y$ & 6 & 3 & 7 & 9 & 10 & 5 & 8 & 4 & 2 & 1\\ \hline
  $7^y$ & 7 & 5 & 2 & 3 & 10 & 4 & 6 & 9 & 8 & 1\\ \hline
  $8^y$ & 8 & 9 & 6 & 4 & 10 & 3 & 2 & 5 & 7 & 1\\ \hline
  $9^y$ & 9 & 4 & 3 & 5 & 1 & 9 & 4 & 3 & 5 & 1\\ \hline
  $10^y$ & 10 & 1 & 10 & 1 & 10 & 1 & 10 & 1 & 10 & 1
\end{tabular}

\item How would you compute $x^{98}$ mod $p$ using repeated squaring. Show
  your work.

First, express $x^{98}$ as:\\\\
$x^{98} = x^{64+32+2} = x^{64} * x^{32} * x^{2}$\\

Next, compute $x^{64}, x^{32}, x^{2}$ using repeated squaring:\\\\
$x^2 = x * x$\\
$x^4 = x^2 * x^2$\\
$x^8 = x^4 * x^4$\\
$x^{16} = x^8 * x^8$\\
$x^{32} = x^{16} * x^{16}$\\
$x^{64} = x^{32} * x^{32}$\\

Therefore, $x^{98} \mod p = x^{64} * x^{32} * x^{2} \mod p$

\end{enumerate}
\end{document}
