%%%%%%%%%%%%%%%%%%%%%%%%%%%%%%%%%%%%%%%%%%%%%%%%%%%%%%%%%%%%%%%%%%%%%%%%%%%
% This is LaTeX file for Homework Assignment 3
% Author: Shuo Yang
%%%%%%%%%%%%%%%%%%%%%%%%%%%%%%%%%%%%%%%%%%%%%%%%%%%%%%%%%%%%%%%%%%%%%%%%%%%

\documentclass[11pt]{article}
\usepackage{amsmath,amssymb,epsfig,graphics,hyperref,amsthm,mathtools}
\DeclarePairedDelimiter\ceil{\lceil}{\rceil}
\DeclarePairedDelimiter\floor{\lfloor}{\rfloor}

\hypersetup{colorlinks=true}

\setlength{\textwidth}{7in}
\setlength{\topmargin}{-0.575in}
\setlength{\textheight}{9.25in}
\setlength{\oddsidemargin}{-.25in}
\setlength{\evensidemargin}{-.25in}

\reversemarginpar
\setlength{\marginparsep}{-15mm}

\newcommand{\rmv}[1]{}
\newcommand{\bemph}[1]{{\bfseries\itshape#1}}
\newcommand{\N}{\mathbb{N}}
\newcommand{\Z}{\mathbb{Z}}
\newcommand{\imply}{\to}
\newcommand{\bic}{\leftrightarrow}

% Some user defined strings for the homework assignment
%
\def\CourseCode{CS566}
\def\AssignmentNo{7}
\def\DateHandedOut{Fall, 2015}
\def\Author{Shuo Yang}
\def\GradeID{50}

\begin{document}

\noindent

\CourseCode \hfill \DateHandedOut

\begin{center}
Homework Assignment \#\AssignmentNo\\
Student: \Author\\
GradeID: \GradeID\\
\end{center}

% A horizontal split line
\hrule\smallskip

\vspace{1.5em}

\begin{enumerate}
\item What are some of the benefits of using Diffie-Hellman over RSA
  for key exchange?

First of all, Diffie-Hellman has performance advantage over RSA
because generating DH keys is cheap, while generating RSA keys is
expensive. Second, DH allows two people who don't know each other to
securely exchange symmetric keys. RSA requires that they know each
other’s public keys. Third, DH is often used for (Perfect) Forward
Secrecy (PFS) since DH key generation is cheap.

\item What is the things RSA can be used to perform whereas
  Diffie-Hellman cannot?

RSA can be used for encryption and decryption while DH is just a key
exchange algorithm. In addition, RSA can also be used for digital
signatures.

\item What is forward secrecy?

A communication protocol has forward secrecy if the compromise of
long-term keys does not compromise past session keys.

\item Let p = 29. Let g = 5. Let Alice’s secret x = 7. Let Bob’s
  secret y = 17. Compute K1. Compute K2. Show your work.

\underline{Alice}:\\
$X = g^x \mod p = 5^7 \mod 29 = 28$\\

\underline{Bob}:\\
$Y = g^y \mod p = 5^{17} \mod 29 = 9$\\

\underline{Alice}:\\
$K_1 = Y^x \mod p = 9^7 \mod 29 = 28$\\

\underline{Bob}:\\
$K_2 = X^y \mod p = 28^{17} \mod 29 = 28$\\

\item Compute $\phi(1343)$. Show your work.

First, identify the prime factors of $1343$: $1343 = 17 *
79$. Therefore $\phi(1343) = (17-1) * (79-1) = 1248$.

\item Compute $3232313123213^{24} \mod 35$. Show your work.

Since $35 = 5 * 7$, $\phi(35) = 4 * 6 = 24$.\\ 
And $3232313123213$ is relatively prime to $35$ since
$GCD(3232313123213, 35) = 1$ because none of $5,7$ or $35$ is a factor
of $3232313123213$.\\  
According to Euler's theorem, $3232313123213^{24} \mod 35 =
3232313123213^{\phi(35)} \mod 35 = 1$.

\item Compute $822981828882^{264} \mod 299$. Show your work.

Since $299 = 13 * 23$, its prime factors are $13$ and $23$. Therefore
$\phi(299) = 12 * 22 = 264$.\\
And $822981828882$ is relatively prime to $299$ since
$GCD(822981828882, 299) = 1$ because none of $13,23$ or $299$ is a
factor of $822981828882$.\\
According to Euler's theorem, $822981828882^{264} \mod 299 =
822981828882^{\phi(299)} \mod 299 = 1$.

\end{enumerate}
\end{document}
